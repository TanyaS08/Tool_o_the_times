\anglais
\doublespacing
\chapter{General Conclusion}\label{Conclusion}
\begin{refsection}

As species interaction networks are determined by ecological and evolutionary mechanisms that have played out across spatial and temporal scales the measures that define their structure (properties) are also capturing information about the processes that have played a role in structuring them. Thus the properties of a network are not only representative of its structure but also a measure of \emph{how} different processes have played a role in determining it, meaning that when we are measuring the property of a network we are also capturing information about the network. This information is something that can be used for making predictions about network structure, \emph{e.g.,} it has been shown that it is possible to to make network-level inferences with very low-level data \cite{MacDonald2020Revisiting, Banville2023What}, or to help make within network predictions to find missing links (\cite{Poisot2023Network, Stock2021Pairwise}). The ability to use a 'simple' measure of the species community for a given location (such as species richness) and to have an estimate of the structure of the potential network (such as connectance) is truly amazing and speaks to how much information we are able to extract either from species networks, or alternatively use to make network predictions. This is something is echoed throughout this thesis.

\section{What we have learnt}

\subsection{Prediction is attainable and feasible}

Chapter~\ref{Foodweb} showcases that with very little 'real world' information we can make accurate predictions by using the information that is 'encoded' in existing interaction networks. This is also echoed in Chapter~\ref{SVD}, where we can see the \emph{immense} amount of information that networks contain and that it is a case of finding ways to access and use this information and, in this instance, using it to make network predictions. It is also clear that this information can go a long way, the metaweb for Canada shared only 4\% of species with its European counterpart, yet we were able to recover 91\% of the interactions. From an ecological perspective this highlights how the laws governing interactions (\emph{sensu} common backbones from \cite{BramonMora2018Identifying}) are being conserved phylogenetically (\cite{Davies2021EcoRed, Elmasri2020HieBay, Gomez2010EcoInt}). From a more practical perspective we show that this transfer learning framework is primed to be adopted as a 'gap-filling' tool as it does not require \emph{extensive} data or computational resources. It is worth noting that in order to predict the Canadian metaweb one only needs to access three different data sources (the species community for Canada, the European metaweb, and a well resolved phylogeny) and that it is possible to execute the required code on a standard laptop, underscoring the lightweight nature of the framework. Finally, the transfer learning framework itself has a lot of scope to be modified should the transfer task have a different set of \emph{e.g.,} data requirements (this is discussed extensively in \autoref{Perspectives}). 

As highlighted by the broad, scoping, discussion in Chapter~\ref{Roadmap} transfer learning is not the only way to approach transfer learning and there are a variety of methodological approaches and data sources that we can tap into when wanting to make network predictions (\autoref{fig:conceptual}. Having a body of work that explicitly addresses the idea of machine learning for network prediction will be particularly valuable as the popularity and interest in alternative 'non-statistical' methods continues to grow within the field of ecology and evolution (\cite{Pichler2023Machine, Cuff2023Chapter}). This chapter also highlights that as our access to the 'auxiliary data' and the computational power we need for network prediction grows we are methodologically and computationally in a prime position to start making feasible network predictions. Hopefully the ideas discussed in these chapters will also allow us to develop even more ''unreasonably effective'' methods for network prediction, thereby providing us with an even more diverse set of approaches we can use for the different scenarios we will inevitably be faced with in the quest to fill in the global gaps.

\subsection{Tools for cross-regional comparison}

Chapter~\ref{SVD} highlights that we need to be critical of the 'tools' we are using when are trying to make cross-region comparisons, and that quantifying the complexity of networks remains, well, complex \cite{Riva2023Cohesive}. Taken at face value it appears that (at least bipartite networks) are extremely complex. All of the 226 networks that we looked at are near maximal 'physical complexity', with all networks being near maximal rank and having an SVD entropy greater than 0.8. However, it is the comparison of SVD entropy with the other (structural) measures of complexity (nestedness, connectance, and spectral radius), that highlights the need for us to be critical of the tools we are using. It is clear that structural measures of 'complexity' are in fact capturing a different facet of network complexity than when we are using SVD entropy (\autoref{fig:other}), this is due to fundamental differences in what aspect of network 'complexity' these measures are trying to capture. One could argue that SVD entropy provides a more fundamentally “correct” measure of complexity as it is quantifying the information within a network as opposed to the number of components/parts a network has and thus should have a place in the toolkit of network descriptors. In addition we show in \autoref{larger-networks-are-less-complex-than-they-could-be} that despite their high complexity networks are still not reaching their highest potential complexity. Although we suggest that the assembly dynamics of networks may play a role, it still raises the question as to why larger networks are not maintaining their complexity and opens the door to questions about how assembly (time) shapes ecological networks (\cite{Barbier2018GenAss, Saravia2018EcoNet}).

One of the biggest challenges we will be faced with as network ecology moves from trying to fill in the global gaps to grappling with global scale questions is that we need to be able to delimit them. The software developed in Chapter~\ref{SpatialBoundaries} is primed for this specific challenge once we begin to leverage global-scale data to understand the spatial structure of networks, especially with regards to being able to discretise them. In this sense this chapter is perhaps the most open-ended component of this thesis, but as such it has a great deal of potential applications, particularly addressing a very simple question - where do networks stop? This is particularly meaningful even in the context of network prediction - at what scale should we be making our network predictions? Although there may be methodological constraints that determine how large the (for example) taxonomic scope of the task of network prediction should be (see \autoref{identifying-the-properties-of-the-network-to-embed}) there is also the question as to what constitutes the correct biogeographic (and socio-political) extent for prediction. Thus being able to 'draw boundaries' around networks has both theoretical as well as applied significance. 

\subsection{Putting it all together}

Arguably the \emph{potential} applicability of the potential applications of \autoref{SpatialBoundaries} represents a culmination of the bulk of the work presented in this thesis. Although Chapters one through three showcase the 'attainability' of network prediction one core aspect that we are still missing is knowing exactly how to delimit the scope (specifically spatial area) of our predictions. In \autoref{Foodweb} we use 'Canada' as the area for prediction, however Canada is a geopolitical unit and there is no strong ecological reason to have omitted the other parts of the North American landmass from the scope of prediction, as there is no strong environmental boundary on the Canada-US border. There is of course also the inverse argument that then questions what would be the optimal 'area' to have made the predictions for \autoref{Foodweb} at - it is not feasible, nor ecologically pertinent, to want to construct a metaweb for The Americas in their entirety. There is thus a need, from a practical perspective, to be able to discretise the area (or species community) that we wish to make a network prediction for, and in order to do that we need to build a theoretical understanding of the boundaries between networks. These ideas are discussed in more depth in \autoref{Defining-ecotrophic-zones}, however it is the functionality of \texttt{SpatialBoundaries.jl} that can facilitate the advancement and development of theory and ideas related to boundaries between networks and help to guide us when choosing the scale at we should be making our predictions at.

\section{The direction moving forward}

\subsection{Scrutinising our methods}

Although chapters~\ref{Roadmap} and \ref{Perspectives} discuss predictive methods (and \autoref{Foodweb} provides a tangible example thereof) the job isn't done when it comes to evaluating the data we are using for prediction. More recent work is showing that the imbalances in current data might be a large problem (especially the false negative rate, \emph{i.e.,} interactions that do occur but are missed in the field, and thus viewed as being 'absent' from the system). When reading the work from \cite{Brimacombe2023Shortcomings}, \cite{Catchen2023Missing}, and \cite{Poisot2023Guidelines} one can't help but to be a bit hesitant to adopt a purely predictive framework, however, as we show in \autoref{rdpg-yields-an-accurate-classifier} the transfer learning model does do quite well, even when we bring ''false interactions'' into the dataset. Of course this does highlight the need to be critical (or at least cautious) when it comes to using datasets for learning, and highlights the need for identifying priority sampling locations and (maybe) even priority interactions, (\emph{e.g.,} \href{https://github.com/EcoJulia/BiodiversityObservationNetworks.jl/tree/main}{\texttt{some}} of the work coming out of the GeoBon group focusing on locating priority sampling locations) to help create a 'best subset' of datasets that can be used for additional data curation.

Even the work presented in \autoref{Foodweb} has room for expansion, and we can (and should) try and push the limits further to see where this transfer learning framework `breaks'. One tempting challenge would be to try and construct a metaweb for Australian mammals --- One is inclined to think that if one were to use the framework from \autoref{Foodweb} `out of the box' the predictions would have a large degree of uncertainty around them due to the taxonomic relationships between Europe and Australian mammals. But this does make for a case study to experiment with other transfer mediums (such as traits). An additional `testing ground' that could prove interesting is to look at rewilding as well as species invasions. Within the context of rewilding one can test how well the predictions are able to 'forecast' the potential impacts of a re-introduced species \emph{a priori} (which one could validate using existing rewilding projects), as well as assess the utility of different candidate ecological surrogates that may be earmarked for introduction in a specific area. The latter point may be particularly useful as one of the main goals is to target the trophic complexity of the area to be rewilded (\cite{Perino2019Rewilding}). For invasions, this can be used to prioritise species that are expected to have a disproportionate impact on the community and flag them in advance as potential threats (\cite{David2017Chapter}).

\subsection{Defining ecotrophic zones}\label{Defining-ecotrophic-zones}

Networks are dynamic, and they can vary across space (\cite{Golubski2016EcoNet, Vazquez2007SpeAbu}) or time (\cite{Poisot2015Species, Trojelsgaard2016EcoNet}) as a function of the environmental conditions. Naturally, we expect network properties to also be dynamic and vary over - in this instance - space. Spatial wombling can be used as a starting point for understanding \emph{how} networks vary across a landscape, particularly if we were to combine this with information on environmental change. \autoref{supp:boundaries} shows some initial (and by no means well resolved) ideas of how we can use the \texttt{SpatialBoundaries.jl} along with the metacommunity model developed by \cite{Thompson2017Dispersal} to look at how the environmental, species community, and network communities boundaries compare within a landscape.

With regards to environmental change it might be interesting to compare species turnover and network changes across the landscape, specifically if we see similar patterns of rates of change at the species or community level and with regards to network structure. This is interesting because there area a myriad of ways we might expect networks to change (or not change) along environmental gradients. Firstly, we might expect network structure to be constant along gradients due to energetic or evolutionary constraints that force netwroks to take on a specific shape, \emph{sensu lato} conserved backbones (\cite{BramonMora2018Identifying}). Alternatively network structure does indeed change along a gradient. This could be due to intraspecific variation that causes the re-wiring of interactions (\cite{Bolnick2011WhyInt}) or changes in species composition are also driving changes in the resulting network (\cite{Martins2022Global}).

\subsubsection{How do the structures within networks vary}

There is also the scope to develop a more nuanced understanding of how
the landscape structures networks, specifically how the different nodes
(\emph{i.e.} species) of the network will perceive the landscape
differently. When looking at other fields of ecology \emph{e.g.,} productivity-diversity studies it is clear that the nature of the relationship between productivity and diversity is scale-dependent (\cite{Chase2002SpaSca, Gillman2006InfPro}). It stands to reason that this will also be the case when looking at interaction networks, specifically how 'boundaries' may be dependent on the node (species). Which means that we might expect \emph{within} network changes \emph{e.g.,} motifs (specific patterns of linkages in a network) to vary across the landscape even if the larger, regional network structure may remain stable. That being said, there is a compelling argument for the need to `combine' these smaller functional units with larger spatial networks (\cite{Fortin2021Network}) and that we should also start thinking about the interplay of time and space (\cite{Estay2023Editorial}). Although deciding exactly what measure might actually be driving differences between local networks and the regional metaweb might not be that simple (\cite{Saravia2022Ecological}).

\subsubsection{Boundaries for policy or management}

Although this section argues for a more theoretical approach to understanding boundaries in the context of potential assembly patterns/constraints there is also a strong argument for being able to draw lines around communities in the context of having a network (or, more realistically, a metaweb) as an 'object' that can be used in a policy or management context. In \autoref{the-metaweb-merges-ecological-hypotheses-and-practices} and \autoref{conclusion-metawebs-predictions-and-people} we briefly mention that the scale of prediction should be ecologically relevant, but should also take into consideration the social aspect of why (and how) we are making predictions. To me, there is an argument that this is also the case when thinking about network boundaries. Given that policy and legislation are enacted at various levels of government or other ruling bodies, being able to identify the boundaries between networks may in fact be a powerful tool at the governance level. Being able to delimit interacting communities (\emph{i.e.,} identify a metaweb) is surely more meaningful than looking solely at species inventories or community composition, particularly if one is truly concerned with conserving ecosystem functions or processes (\cite{Wood2022Missing, ThuillerNavigating}). However, I feel it is important to stress that the idea of trying to draw boundaries should be approached with caution and sensitivity, especially within in the context of 'doing no harm' (\emph{sensu} Box 2 of section~\ref{conclusion-metawebs-predictions-and-people}) and understanding that ecological and socio-political 'boundaries' may in fact have different 'goals' or contexts.

\subsection{The future collaborative toolbox}

On a more contemplative note, I want to discuss the value of thinking about the development of further tools for the toolbox analogy used in this thesis. A significant amount of work in this thesis was only possible with the support and intellectual contribution of many collaborators and there is an argument for continuing this strong network of collaboration for the development of future such tools. From a purely practical perspective the continued push for developing biology-centric packages within the \texttt{Julia} language (\cite{Roesch2021Julia}) requires that we maintain interoperability between packages and build a collection of tools that build on and fit in amongst each other. Looking at the science/theoretical side of the toolbox, a unified idea or goal for moving the macroecological network 'agenda' forward means that we can build on ideas and thoughts in a more cohesive manner. For example (\cite{Dansereau2023Spatially, Catchen2023Improving, Banville2023What}) have all already used the work presented in this thesis to take the ideas discussed in new and further directions. This is not to say that we should not also work on developing 'competing' methods (although I would argue 'competing' here is used in the context for finding alternative approaches to solving a similar problem \emph{e.g.,} \cite{Caron2022Addressing} takes a more trait-based approach to network prediction), but there is strong evidence that in working together we can get where we want to be sooner. The 'toolbox' that this thesis represents is but a small step in thinking about interaction networks at a global scale, but it is nevertheless an important step, as it will hopefully lay the groundwork for even more innovation and creation.

\printbibliography{}
\end{refsection}

\endinput